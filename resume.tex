%-------------------------
% Resume in Latex
% Author : Pratyay Gopal
%------------------------

\documentclass[letterpaper,11pt]{article}

\usepackage{latexsym}
\usepackage[empty]{fullpage}
\usepackage{titlesec}
\usepackage{marvosym}
\usepackage[usenames,dvipsnames]{color}
\usepackage{verbatim}
\usepackage{enumitem}
\usepackage[hidelinks]{hyperref}
\usepackage{fancyhdr}
\usepackage[english]{babel}
\usepackage{tabularx}
\input{glyphtounicode}


%----------FONT OPTIONS----------
% sans-serif
% \usepackage[sfdefault]{FiraSans}
% \usepackage[sfdefault]{roboto}
% \usepackage[sfdefault]{noto-sans}
% \usepackage[default]{sourcesanspro}

% serif
% \usepackage{CormorantGaramond}
% \usepackage{charter}


\pagestyle{fancy}
\fancyhf{} % clear all header and footer fields
\fancyfoot{}
\renewcommand{\headrulewidth}{0pt}
\renewcommand{\footrulewidth}{0pt}

% Adjust margins
\addtolength{\oddsidemargin}{-0.5in}
\addtolength{\evensidemargin}{-0.5in}
\addtolength{\textwidth}{1in}
\addtolength{\topmargin}{-.5in}
\addtolength{\textheight}{1.0in}

\urlstyle{same}

\raggedbottom
\raggedright
\setlength{\tabcolsep}{0in}

% Sections formatting
\titleformat{\section}{
  \vspace{-4pt}\scshape\raggedright\large
}{}{0em}{}[\color{black}\titlerule \vspace{-5pt}]

% Ensure that generate pdf is machine readable/ATS parsable
\pdfgentounicode=1

%-------------------------
% Custom commands
\newcommand{\resumeItem}[1]{
  \item\small{
    {#1 \vspace{-2pt}}
  }
}

\newcommand{\resumeSubheading}[4]{
  \vspace{-2pt}\item
    \begin{tabular*}{0.97\textwidth}[t]{l@{\extracolsep{\fill}}r}
      \textbf{#1} & #2 \\
      \textit{\small#3} & \textit{\small #4} \\
    \end{tabular*}\vspace{-7pt}
}

\newcommand{\resumeSubSubheading}[2]{
    \item
    \begin{tabular*}{0.97\textwidth}{l@{\extracolsep{\fill}}r}
      \textit{\small#1} & \textit{\small #2} \\
    \end{tabular*}\vspace{-7pt}
}

\newcommand{\resumeProjectHeading}[2]{
    \item
    \begin{tabular*}{0.97\textwidth}{l@{\extracolsep{\fill}}r}
      \small#1 & #2 \\
    \end{tabular*}\vspace{-7pt}
}

\newcommand{\resumeSubItem}[1]{\resumeItem{#1}\vspace{-4pt}}

\renewcommand\labelitemii{$\vcenter{\hbox{\tiny$\bullet$}}$}

\newcommand{\resumeSubHeadingListStart}{\begin{itemize}[leftmargin=0.15in, label={}]}
\newcommand{\resumeSubHeadingListEnd}{\end{itemize}}
\newcommand{\resumeItemListStart}{\begin{itemize}}
\newcommand{\resumeItemListEnd}{\end{itemize}\vspace{-5pt}}

%-------------------------------------------
%%%%%%  RESUME STARTS HERE  %%%%%%%%%%%%%%%%%%%%%%%%%%%%


\begin{document}

%----------HEADING----------

\begin{flushleft}
    \textbf{\huge Pratyay Rudravaram} \\ \vspace{1pt}
    \underline{+1 945-268-0041} $|$ 
    \href{mailto:pratyay2@illinois.edu}{\underline{pratyay2@illinois.edu}} $|$ 
    \href{https://linkedin.com/in/pratyay-gopal}{\underline{linkedin.com/pratyay-gopal}} $|$
    \href{https://pratyaygopal.github.io/}{\underline{pratyaygopal.github.io}} $|$
    \small US Citizen  
\end{flushleft}

    \vspace{-20pt}



%-----------EDUCATION-----------
\section{Education}
  \resumeSubHeadingListStart
    \resumeSubheading
      {University Of Illinois Urbana Champaign}{GPA: 3.66/4.00}
      {Bachelor of Science in Computer Engineering, Minor in Mathematics - James Scholar Honors}{December 2026}
    
  \resumeSubHeadingListEnd
    \begin{itemize}
    \setlength\itemsep{-5pt}
    \item \textbf{Relevant Coursework} : Computer Organization and Design, FPGA, Operating Systems, Digital Design, Logic Synthesis, Parallel Programming, Data Structures, Electronics, Analog Signal Processing
    \end{itemize}

    \vspace{-15pt}

%--------UNI STUFF--------------------
\section{University Experience}
  \resumeSubHeadingListStart
    
    \resumeSubheading
      {Undergraduate Teaching Assistant, ECE385(Digital Systems and FPGA)}{Jan 2025 -- Present}
      {}{}
      \vspace{-18pt}
      \resumeItemListStart
        \resumeItem{Hosted office hours for course’s FPGA projects including VGA text controller and RISC processor.}
        \resumeItem{Moderated the 1000+ members class discord server and clarified student questions on SV and FPGA testing.}
        \resumeItem{Conducted demos, reviewed SystemVerilog testbenches, and facilitated hardware debugging and verification.}
      \resumeItemListEnd
      
    \resumeSubheading
      {Officer, ACM SIGARCH@UIUC }{Aug 2024 -- Present}
      {}{}
      \vspace{-18pt}
      \resumeItemListStart
        \resumeItem{Officer and Workshop Lead for UIUC’s premier computer architecture student organization. }
        \resumeItem{Designing workshops to introduce students to RTL design and simulation, computer architecture and ISAs.}
        \resumeItem{Organized meetings to discuss papers on topics like transactional memory, branch prediction, and cache coherence.}
      \resumeItemListEnd
  \resumeSubHeadingListEnd

%-----------EXPERIENCE-----------
\section{Work Experience}
  \resumeSubHeadingListStart
    \resumeSubheading
      {RTL Verification Intern, Deepgrid Semi}{June 2025 -- July 2025}
      {}{}
      \vspace{-18pt}
      \resumeItemListStart
        \resumeItem{Developed a testing strategy to verify 5 modules of an open-source hardware accelerator using executed SystemVerilog UVM-esqe testbenches while ensuring 100\% functional correctness and simulation coverage.}
        \resumeItem{Collaborated on the testing of a RISC-V based image detection system targeting the Arty A7 FPGA platform.}
      \resumeItemListEnd

    \resumeSubheading
      {PCB Design Intern, Apollo Computing Laboratories}{June 2024 -- Aug 2024}
      {}{}
      \vspace{-18pt}
      \resumeItemListStart
        \resumeItem{Developed a 2-layer 8-channel power delivery board to enable high availability data-center applications.}
        \resumeItem{Designed and implemented hardware to control power supply to a  server, ensuring precise control.}
      \resumeItemListEnd

    \resumeSubheading
      {Electronics Intern, Thermo Fisher Scientific}{May 2022 – July 2022}
      {}{}
      \vspace{-18pt}
      \resumeItemListStart
        \resumeItem{Coded a Raspberry pi test-bench to calibrate a thermocouple for blood bank temperature regulation using python.}
        \resumeItem{Simulator was deployed to 10+ locations across India and was used to calibrate blood bank coolers for a lower cost}
    \resumeItemListEnd

  \resumeSubHeadingListEnd


%-----------PROJECTS-----------
\section{Projects}
    \resumeSubHeadingListStart
        \resumeProjectHeading
          {\textbf{Superscalar Out-of-Order RISC-V CPU} $|$ \emph{SystemVerilog, Verdi, VCS}}{}
          \resumeItemListStart
            \resumeItem{Created a speculative out-of-order RISC-V CPU with an ERR architecture, implementing the RV32IM spec.}
            \resumeItem{ Supports upto 2 instruction commits per cycle, multiple integer execution units, parametric multiplier/div, etc.}
            \resumeItem{Synthesized dual-issue/commit core with L0+L1 cache on FreePDK’s 45nm process node at 525MHz}
          \resumeItemListEnd
        \resumeProjectHeading
          {\textbf{FPGA-Based Video Game – Spartan-7 Board} $|${\fontfamily{qcr} \small \href{https://pratyaygopal.github.io/docs/fnaf.pdf}{pratyaygopal.github.io/docs/fnaf.pdf}}}{}
          \resumeItemListStart
            \resumeItem{Developed a modified port of Five Nights at Freddy’s on a Spartan-7 FPGA, achieving real-time gameplay.}
            \resumeItem{Implemented and integrated an SPI-based keyboard interface supporting up to six simultaneous key presses.}
            \resumeItem{Designed game logic, randomized seed selection and optimized scalable graphics within a 270 KB RAM constraint.}
            \resumeItem{Recognized as one of the top 10 projects in ECE 385 and showcased for its technical complexity.}
          \resumeItemListEnd
        \resumeProjectHeading
          {\textbf{Breadboard Synth} $|$ \emph{Falstad, Arduino, Oscilloscopes}}{}
          \resumeItemListStart
            \resumeItem{Designed and Implemented 4 modules for a fully modular music synthesizer for the ECE198 Honors lab.}
            \resumeItem{Used Falstad for design and worked with electrical workbench tools for testing.}
          \resumeItemListEnd
        \resumeProjectHeading
          {\textbf{Temperature Regulation Calibrator} $|${\fontfamily{qcr} \small \href{https://github.com/pratyaygopal/Thermocouple-Simulator}{github.com/pratyaygopal/Thermocouple-Simulator}}}{}
          \resumeItemListStart
            \resumeItem{Built a proof of concept of an affordable, temperature calibrator using thermocouples and Raspberry Pi.}
            \resumeItem{Ran 100+ simulations of the ADC/DAC and recorded data to ensure minimal variance in the output voltage.}
          \resumeItemListEnd

    \resumeSubHeadingListEnd


    %-----------PROGRAMMING SKILLS-----------
    \section{Technical Skills}
     \begin{itemize}[leftmargin=0.15in, label={}]
        \small{\item{
        \textbf{Languages}{: SystemVerilog, Verilog, Bash, C, C++, VHDL, Python, Java}\\ 
        \vspace{2pt}
        \textbf{Tools}{: Git, KICAD, ORCAD, PADS, Intel Quartus, Xilinx Vivado, VS Code, Verdi, VCS, Verilator} \\
        \vspace{2pt}
        \textbf{Protocols}{: AXI-4, AXI Stream, SPI, TCP, UDP, UART, I2C} \\
        }}
     \end{itemize}

    \vspace{-15pt}

    

%-------------------------------------------
\end{document}